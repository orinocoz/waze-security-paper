Waze es una aplicación social para conductores donde los usuarios pueden buscar direcciones y Waze los guiará calculando una ruta de acuerdo a las preferencias del usuario y las condiciones de los posibles caminos. Las condiciones del camino son determinada por lo que los mismos usuarios retroalimentan al usar la aplicación mientras conducen y la aplicación en compensación por ese feedback, otorga puntaje e íconos personalizados. Los usuarios más experimentados incluso pueden editar el mapa para contribuir en posibles problemas que existan como por ejemplo vías en sentido contrario o los mismos cambios contemporáneos de construcciones de nuevos caminos. También los usuarios pueden alertar la presencia de policías, problemas en el camino, accidentes, problemas de tráfico y por último la instancia de poder conversar (chat) o enviar mensajes a otros “wazers”.
\\\\
En este informe, auditaremos Waze analizando el tráfico que genera en cada una de sus acciones con el objetivo de encontrar vulnerabilidades por alguna mala práctica de los programadores de la aplicación o bien de los sistemas. 
\\\\
Fue elegida Waze porque es una aplicación con muchos usuarios, con funciones importantes que ayudan a una ciudad más inteligente, además de un chat entre los usuarios con lo que si hubiera una posibilidad de ataque, se podrían producir cambios drásticos en las vías escogidas, generando problemas de tráfico en alguna avenida o carretera importante en algún país del mundo provocando retrasos en las llegadas de muchas personas a sus destinos que podría incidir económicamente en dicho país. Además de ser una de las pocas aplicaciones del ámbito, ya que de por sí, es un elemento distractor.
\\\\
Un caso práctico en Chile podría ser en el área del cobre, al ser un pilar importante en la economía nacional. Si este ataque se hiciera en la Región de Antofagasta, muchos trabajadores verán retrasada su llegada a la mina de Chuquicamata. Lo mismo que en el caso de la Región de O'Higgins para la mina El teniente, entre otras. Que baje la producción de cobre en Chile afecta negativamente la economía nacional, incluso en millones de dólares en pérdidas por falta de mano de obra.