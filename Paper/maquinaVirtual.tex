En VirtualBox, aunque se instalamos más de una aplicación como fake gps (figura 17):


        \begin{figure}[H]
  \begin{center}
    \includegraphics[width=0.6\textwidth]{imagenes/fig31.png}
    \caption{Aplicaciones instaladas para fakegps}
  \end{center}
\end{figure}


Waze no mostró el mapa, como se ve en la figura 20:

        \begin{figure}[H]
  \begin{center}
    \includegraphics[width=0.6\textwidth]{imagenes/fig32.png}
    \caption{Waze no muestra mapa en maquina virtual (VirtualBox) }
  \end{center}
\end{figure}

En la máquina virtual al enviar una alerta, no aparece el mensaje de alerta enviada, figura 19:

        \begin{figure}[H]
  \begin{center}
    \includegraphics[width=0.6\textwidth]{imagenes/fig33.png}
    \caption{En la maquina virtual no se muestra el mensaje de alerta enviada}
  \end{center}
\end{figure}

Como es el caso de enviar una alerta desde un celular:

        \begin{figure}[H]
  \begin{center}
    \includegraphics[width=0.3\textwidth]{imagenes/fig34.png}
    \caption{Mensaje de alerta enviada mostrada en un celular}
  \end{center}
\end{figure}

Otra opción descartando VirtualBox, fue probar con el emulador de android BlueStacks:

        \begin{figure}[H]
  \begin{center}
    \includegraphics[width=0.7\textwidth]{imagenes/fig35.png}
    \caption{Waze en el emulador de Android Bluestacks}
  \end{center}
\end{figure}

En este caso si se muestra el mapa, y también es posible enviar una alerta:

        \begin{figure}[H]
  \begin{center}
    \includegraphics[width=0.7\textwidth]{imagenes/fig36.png}
    \caption{En Bluestack, Waze si muestra el mapa como en un celular}
  \end{center}
\end{figure}


y luego de enviarla aparece en el celular:

        \begin{figure}[H]
  \begin{center}
    \includegraphics[width=0.3\textwidth]{imagenes/fig37.png}
    \caption{Alerta enviada usando Bluestacks aparece se muestra en un celular}
  \end{center}
\end{figure}

El problema que encontramos con BlueStacks fue que no es posible acceder a los archivos de la aplicación desde afuera. Pero si en la máquina virtual de Android:

        \begin{figure}[H]
  \begin{center}
    \includegraphics[width=0.7\textwidth]{imagenes/fig38.png}
    \caption{Archivos de Waze dentro de la maquina virtual}
  \end{center}
\end{figure}

        \begin{figure}[H]
  \begin{center}
    \includegraphics[width=0.7\textwidth]{imagenes/fig39.png}
    \caption{Más archivos de Waze dentro de la máquina virtual}
  \end{center}
\end{figure}

Dentro de los archivos no se encontró nada configurable. El archivo lang.conf solo mantenía los nombres de los lenguajes disponibles del software. [Figura 26 y 27]

En los archivos menús al parecer mantiene llamadas a funciones de Waze, ya que solo mantiene líneas con nombres de acciones aparentemente. [Figura 26]

        \begin{figure}[H]
  \begin{center}
    \includegraphics[width=0.9\textwidth]{imagenes/fig40.png}
    \caption{Contenido archivo alerts.menu}
  \end{center}
\end{figure}

De todas formas, intentamos modificar para ver si esto tiene incidencias en la aplicación, modificando el mismo archivo de alertas con uno similar:


        \begin{figure}[H]
  \begin{center}
    \includegraphics[width=0.6\textwidth]{imagenes/fig41.png}
    \caption{Archivo alerts.menu modificado}
  \end{center}
\end{figure}

Pero esto no tuvo repercusiones en la aplicación:


        \begin{figure}[H]
  \begin{center}
    \includegraphics[width=0.6\textwidth]{imagenes/fig42.png}
    \caption{Menu de alertas no es afectado por la modificación del archivo alerts.menu}
  \end{center}
\end{figure}


Se volvió a verificar la aplicación y esta había reemplazado el archivo alerts.menu con el archivo original, por tanto, podemos deducir que la aplicación genera estos ficheros al iniciar.

